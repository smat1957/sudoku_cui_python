% latex uft-8
\documentclass[uplatex,a4paper,11pt,oneside,openany]{jsbook}
%
\usepackage[dvipdfmx]{graphicx}
\usepackage{amsmath,amssymb}
\usepackage{bm}
\usepackage{graphicx}
\usepackage{ascmac}
\usepackage{setspace}
\usepackage{here}
\usepackage{listings,jlisting} %日本語のコメントアウトをする場合jlistingが必要
%ここからソースコードの表示に関する設定
\usepackage{xcolor,comment}

\definecolor{mygreen}{rgb}{0,0.6,0}
\definecolor{mygray}{rgb}{0.5,0.5,0.5}
\definecolor{mymauve}{rgb}{0.58,0,0.82}

\begin{comment}
\lstset{
  backgroundcolor=\color{white},   % choose the background color; you must add \usepackage{color} or \usepackage{xcolor}; should come as last argument
  basicstyle=\footnotesize,        % the size of the fonts that are used for the code
  breakatwhitespace=false,         % sets if automatic breaks should only happen at whitespace
  breaklines=true,                 % sets automatic line breaking
  captionpos=b,                    % sets the caption-position to bottom
  commentstyle=\color{mygreen},    % comment style
  deletekeywords={...},            % if you want to delete keywords from the given language
  escapeinside={\%*}{*)},          % if you want to add LaTeX within your code
  extendedchars=true,              % lets you use non-ASCII characters; for 8-bits encodings only, does not work with UTF-8
  firstnumber=1000,                % start line enumeration with line 1000
  frame=single,	                   % adds a frame around the code
  keepspaces=true,                 % keeps spaces in text, useful for keeping indentation of code (possibly needs columns=flexible)
  keywordstyle=\color{blue},       % keyword style
  language=Octave,                 % the language of the code
  morekeywords={*,...},            % if you want to add more keywords to the set
  numbers=left,                    % where to put the line-numbers; possible values are (none, left, right)
  numbersep=5pt,                   % how far the line-numbers are from the code
  numberstyle=\tiny\color{mygray}, % the style that is used for the line-numbers
  rulecolor=\color{black},         % if not set, the frame-color may be changed on line-breaks within not-black text (e.g. comments (green here))
  showspaces=false,                % show spaces everywhere adding particular underscores; it overrides 'showstringspaces'
  showstringspaces=false,          % underline spaces within strings only
  showtabs=false,                  % show tabs within strings adding particular underscores
  stepnumber=2,                    % the step between two line-numbers. If it's 1, each line will be numbered
  stringstyle=\color{mymauve},     % string literal style
  tabsize=2,	                   % sets default tabsize to 2 spaces
  title=\lstname                   % show the filename of files included with \lstinputlisting; also try caption instead of title
}
\end{comment}

%ここからソースコードの表示に関する設定

\lstdefinestyle{customc}{
  belowcaptionskip=1\baselineskip,
  breaklines=true,
  numbers=left,
  frame=single,
  xleftmargin=\parindent,
  language=C,
  showstringspaces=false,
  basicstyle=\footnotesize\ttfamily,
  keywordstyle=\bfseries\color{green!40!black},
  commentstyle=\itshape\color{purple!40!black},
  identifierstyle=\color{blue},
  stringstyle=\color{orange},
}

\lstdefinestyle{customjava}{
  belowcaptionskip=1\baselineskip,
  breaklines=true,
  numbers=left,
  frame=single,
  xleftmargin=\parindent,
  language=java,
  showstringspaces=false,
  basicstyle=\footnotesize\ttfamily,
  keywordstyle=\bfseries\color{green!40!black},
  commentstyle=\itshape\color{purple!40!black},
  identifierstyle=\color{blue},
  stringstyle=\color{orange},
}

\lstdefinestyle{custompython}{
  belowcaptionskip=1\baselineskip,
  breaklines=true,
  numbers=left,
  frame=single,
  xleftmargin=\parindent,
  language=Python,
  showstringspaces=false,
  basicstyle=\footnotesize\ttfamily,
  keywordstyle=\bfseries\color{green!40!black},
  commentstyle=\itshape\color{purple!40!black},
  identifierstyle=\color{blue},
  stringstyle=\color{orange},
}

\lstdefinestyle{customasm}{
  belowcaptionskip=1\baselineskip,
  frame=L,
  xleftmargin=\parindent,
  language=[x86masm]Assembler,
  basicstyle=\footnotesize\ttfamily,
  commentstyle=\itshape\color{purple!40!black},
}

\lstset{escapechar=@,style=custompython}

\makeatletter
\def\ps@plainfoot{%
  \let\@mkboth\@gobbletwo
  \let\@oddhead\@empty
  \def\@oddfoot{\normalfont\hfil-- \thepage\ --\hfil}%
  \let\@evenhead\@empty
  \let\@evenfoot\@oddfoot}
  \let\ps@plain\ps@plainfoot
\renewcommand{\chapter}{%
  \if@openright\cleardoublepage\else\clearpage\fi
  \global\@topnum\z@
  \secdef\@chapter\@schapter}
\makeatother
%
\newcommand{\maru}[1]{{\ooalign{%
\hfil\hbox{$\bigcirc$}\hfil\crcr%
\hfil\hbox{#1}\hfil}}}
%
\setlength{\textwidth}{\fullwidth}
\setlength{\textheight}{40\baselineskip}
\addtolength{\textheight}{\topskip}
\setlength{\voffset}{-0.55in}
%
\begin{document}
% START DOCUMENT
%
% COVER
\begin{center}
  \huge \par
  \vspace{15mm}
  \huge \par
  \vspace{15mm}
  \LARGE Sudoku = Number Place (CUI - Python) \par
  \vspace{100mm}
  \Large \date \par
  \vspace{15mm}
  \Large \par
  \vspace{10mm}
  \Large S.Matoike \par
  \vspace{10mm}
\end{center}
\thispagestyle{empty}
\clearpage
\addtocounter{page}{-1}
\newpage
\setcounter{tocdepth}{3}
%
\tableofcontents
%
\chapter{PythonによるSudoku}

Sudokuは、アメリカのデルマガジンズ社が1979年以降にNumber Placeの名前で販売していた数字パズルを、
株式会社ニコリの創業者・社長である鍜治真起氏が「数独」と命名し、1986年に月刊ニコリストで紹介した。
2005年にはSudokuの名で英国で大流行し、世界へ広まった。(英辞郎より)\\

このプログラムは、短い簡易な作りになっているのが魅力的な所です

\subsection{盤面を表示する}

まず、boardリストに盤面を定義し、盤面の印刷を行う関数print\_board()を定義します

\begin{lstlisting}[caption=board,label=sudoku01]
board = [[5,3,0,0,7,0,0,0,0],\
         [6,0,0,1,9,5,0,0,0],\
         [0,9,8,0,0,0,0,6,0],\
         [8,0,0,0,6,0,0,0,3],\
         [4,0,0,8,0,3,0,0,1],\
         [7,0,0,0,2,0,0,0,6],\
         [0,6,0,0,0,0,2,8,0],\
         [0,0,0,4,1,9,0,0,5],\
         [0,0,0,0,8,0,0,7,9]]

def print_board():
    global board
    for y in range(9):
        for x in range(9):
            print(' ',end='')
            if x in [2,5]:
                print(board[y][x], end=' |')
            else:
                print(board[y][x], end=' ')
        if y in [2,5]:
            print('\n---------|---------|--------')
        else:
            print()

\end{lstlisting}

print\_board()をそのまま呼び出して、動作を確認してみます

\begin{verbatim}
# 動作確認
print_board()

  5  3  0 | 0  7  0 | 0  0  0
  6  0  0 | 1  9  5 | 0  0  0
  0  9  8 | 0  0  0 | 0  6  0
 ---------|---------|--------
  8  0  0 | 0  6  0 | 0  0  3
  4  0  0 | 8  0  3 | 0  0  1
  7  0  0 | 0  2  0 | 0  0  6
 ---------|---------|--------
  0  6  0 | 0  0  0 | 2  8  0
  0  0  0 | 4  1  9 | 0  0  5
  0  0  0 | 0  8  0 | 0  7  9
\end{verbatim}

\subsection{数値を置けるか判定する}

次に、引数yとxで指定されたスロットに、
第三引数で受け取ったnを置く事ができるか否かを判定する関数possible()を定義します

(1) 縦一列の中に、nと同じ数字があったら置けませんので、Falseを返します

(2) 横一行の中に、nと同じ数字があったら置けませんので、Falseを返します

(3) 3$\times$3の枠の中に、nと同じ数字があったら置けませんので、Falseを返します

上記(1)、(2)、(3)の何れでもないなら、Trueを返します

\begin{lstlisting}[caption=possible,label=sudoku02]
def possible(y,x,n):
    global board
    for i in range(0,9):
        if board[y][i] == n:
            return False
    for i in range(0,9):
        if board[i][x] == n:
            return False
    x0 = (x//3)*3
    y0 = (y//3)*3
    for i in range(0,3):
        for j in range(0,3):
            if board[y0+i][x0+j] == n:
                return False
    return True

\end{lstlisting}

5行5列目(boardリストの0行目や0列目から数え始めるので、引数はx=y=4)の空きスロットに値を指定して、possible()の動作を確認してみます

\begin{verbatim}
# 動作確認
print( possible(4,4,4) )
print( possible(4,4,5) )
\end{verbatim}

\subsection{問題を解く}

最後に、問題を解く関数solve()を定義します

y行x列目のスロットに着目して、そこが0ならば空きスロットですから、
そのスロットに、1から10までの数字を順に指定して、possible()を呼びだしてはチェックしていきます

もし、possible()関数がTrueを返したら、それは一つの候補ですので、
引き続きsolve()を繰り返して空きスロットを埋めていきます

solve()がreturnでNoneを返したときは、選ばれた候補は使えなかったという事ですので、
盤面のスロットを空(0)に戻しています

全ての空欄が埋まったなら、print\_board()を呼んで結果(答え)を印刷しています

\begin{lstlisting}[caption=solve,label=sudoku03]
def solve():
    global board
    for y in range(9):
        for x in range(9):
            if board[y][x] == 0:
                for n in range(1,10):
                    if possible(y,x,n):
                        board[y][x] = n
                        solve()
                        board[y][x] = 0
                return
    print_board()

\end{lstlisting}

solve()関数の中から、自分自身であるsolve()関数を呼び出す方法、
再帰呼び出し、を使うことによってプログラムを短く記述できています

\begin{verbatim}
# 動作確認
solve()

  5  3  4 | 6  7  8 | 9  1  2
  6  7  2 | 1  9  5 | 3  4  8
  1  9  8 | 3  4  2 | 5  6  7
 ---------|---------|--------
  8  5  9 | 7  6  1 | 4  2  3
  4  2  6 | 8  5  3 | 7  9  1
  7  1  3 | 9  2  4 | 8  5  6
 ---------|---------|--------
  9  6  1 | 5  3  7 | 2  8  4
  2  8  7 | 4  1  9 | 6  3  5
  3  4  5 | 2  8  6 | 1  7  9
\end{verbatim}

%\section*{謝辞}
%\addcontentsline{toc}{chapter}{謝辞}
%
\begin{thebibliography}{99}
  \bibitem{1}
\end{thebibliography}
%
% END DOCUMENT
\end{document}
%
